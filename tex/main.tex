%
% File emnlp2020.tex
%
%% Based on the style files for ACL 2020, which were
%% Based on the style files for ACL 2018, NAACL 2018/19, which were
%% Based on the style files for ACL-2015, with some improvements
%%  taken from the NAACL-2016 style
%% Based on the style files for ACL-2014, which were, in turn,
%% based on ACL-2013, ACL-2012, ACL-2011, ACL-2010, ACL-IJCNLP-2009,
%% EACL-2009, IJCNLP-2008...
%% Based on the style files for EACL 2006 by 
%%e.agirre@ehu.es or Sergi.Balari@uab.es
%% and that of ACL 08 by Joakim Nivre and Noah Smith

\documentclass[11pt,a4paper]{article}
\usepackage[hyperref]{emnlp2020}
\usepackage{times}
\usepackage{latexsym}
\renewcommand{\UrlFont}{\ttfamily\small}
\usepackage{float}
% This is not strictly necessary, and may be commented out,
% but it will improve the layout of the manuscript,
% and will typically save some space.
\usepackage{microtype}

%\aclfinalcopy % Uncomment this line for the final submission
%\def\aclpaperid{***} %  Enter the acl Paper ID here

%\setlength\titlebox{5cm}
% You can expand the titlebox if you need extra space
% to show all the authors. Please do not make the titlebox
% smaller than 5cm (the original size); we will check this
% in the camera-ready version and ask you to change it back.

% START custom header
\usepackage{amsfonts}
\usepackage{bm}
\usepackage{array,amsmath}
\usepackage{amssymb}
\usepackage{dsfont}
\usepackage{float}
\usepackage{graphicx}
\usepackage{algorithm}
\usepackage{algorithmicx}
\usepackage{algpseudocode}
\usepackage{pgf,tikz}
\usepackage{mathrsfs}
\usepackage{multirow}
\usepackage{booktabs}
\usetikzlibrary{arrows}
\usepackage{threeparttable}

\def\figref#1{Figure~\ref{fig:#1}}
\def\figlabel#1{\label{fig:#1}\label{p:#1}}
\def\chapref#1{Chapter~\ref{chap:#1}}
\def\chaplabel#1{\label{chap:#1}\label{p:#1}}
\def\Tabref#1{Table~\ref{tab:#1}}
\def\tabref#1{Table~\ref{tab:#1}}
\def\tablabel#1{\label{tab:#1}\label{p:#1}}
\def\Secref#1{\S\ref{sec:#1}}
\def\secref#1{\S\ref{sec:#1}}
\def\seclabel#1{\label{sec:#1}}
\def\eqref#1{Eq.~\ref{eqn:#1}}
\def\eqrefn#1{\ref{eqn:#1}}
\def\eqsref#1#2{Eqs.~\ref{eqn:#1}/\ref{eqn:#2}}
\def\eqlabel#1{\label{eqn:#1}}
\def\subsp#1{P_{\mbox{{\scriptsize\rm #1}}}}

\def\numpar{100k}
\def\ppnumpar{5}

\newcounter{notecounter}
\newcommand{\enotesoff}{\long\gdef\enote##1##2{}}
\newcommand{\enoteson}{\long\gdef\enote##1##2{{
			\stepcounter{notecounter}
			{\large\bf
				\hspace{1cm}\arabic{notecounter} $<<<$ ##1: ##2
				$>>>$\hspace{1cm}}}}}
\enoteson
%\enotesoff
\long\def\eat#1{}

\def\dnrmupmath#1#2{\mbox{$^#2_{\hbox{\scriptsize #1}}$}}
\def\dnrm#1{\mbox{$_{\hbox{\scriptsize #1}}$}}
\def\uprm#1{\mbox{$^{\hbox{\scriptsize #1}}$}}
\def\cupequal{\cup\!\!=}
\newcommand{\pluseq}{\mathrel{+}=}


% END custom header

\newcommand\BibTeX{B\textsc{ib}\TeX}

\title{Instructions for EMNLP 2020 Proceedings}

\author{First Author \\
  Affiliation / Address line 1 \\
  Affiliation / Address line 2 \\
  Affiliation / Address line 3 \\
  \texttt{email@domain} \\\And
  Second Author \\
  Affiliation / Address line 1 \\
  Affiliation / Address line 2 \\
  Affiliation / Address line 3 \\
  \texttt{email@domain} \\}

\date{}

\begin{document}
\maketitle
\enote{pd}{possible titles? Debiasing BERT with DensRay. Removing Gender Information in Pretrained Language Models with Analytical Solutions. ...}
\begin{abstract}
As one of the representatives of context word embedding, BERT has achieved the most advanced performance on many NLP tasks. Due to the strong feature extraction ability and the high demand for the amounts of training data, BERT can hardly avoid learning many of the human-generated stereotypes in the text data, including gender bias. In this paper, we (1) propose a template based method that is well suited to quantify gender bias in language models; (2) adapt DensRay (a vector space projection analysis method) to contextualized embeddings, and use this method to eliminate gender information. (3) investigate how English training data can be used to remove gender bias in Chinese using multilingual BERT.
\end{abstract}

\section{Introduction}
Word embeddings, which represent the semantic meaning of text data as vectors, are used as input in natural language processing tasks (Goldberg, 2017). It has disclosed that word embeddings exhibit unexpected social biases, such as gender bias, present in their training corpora \citep{bolukbasi2016man, caliskan2017semantics,garg2018word}. An example is that man is associated with computer programmer on the embedding space, and woman is associated with homemaker \citep{bolukbasi2016man}. Contextual word embedding models, such as BERT \citep{devlin2018bert}, have become increasingly common and achieved new state-of-the-art results in the many NLP tasks. Researches have also found gender bias in contextualized embeddings \citep{zhao2019gender,may2019measuring}.

In this work, we aim to mitigate gender bias on BERT embedding in a straight-forward and interpretable way. We introduce a debiasing method on BERT using the DensRay \citep{dufter2019analytical}, which is a computational method to get the interpretable dimensions by rotating the word embedding spaces. We show that gender information is captured in every BERT layer. We applied DensRay to every BERT layer and evaluated on a set of simple templates we constructed and the Word Embedding Association Test (WEAT) \citep{caliskan2017semantics}, our experiments find that the DensRay debiasing method effectively mitigates gender bias, and do little harm to the performance of the BERT model on language modeling and GLUE tasks \citep{wang2018glue}. As an extension, we also applied this debiasing method to the multilingual-BERT (mBERT) model: we use English gender label for computing the rotation matrix, and debias on our Chinese templates. Our contributions are summarized as the following: 

• We introduce the DensRay debiasing method on BERT, and demonstrates the debiasing effectiveness by our templates and the Word Embedding Association Test.

• We show that the DensRay debiasing method can be applied to mBERT for zero-shot debiasing for other languages.

\enote{pd}{I would call this section ``Background''}
\section{Related Work}
\subsection{Quantifying Gender Bias}
A typical way to measure gender bias is to evaluate on
\textbf{downstream tasks}. For coreference resolution,
\cite{zhao2018gender} designed Winobias and
\cite{rudinger2018gender} designed Winogender schemas. In
contrast to WinoBias, Winogender schemas include
gender-neutral pronouns. One Winogender schema has one
occupational mention and one ``other participant'' mention
while WinoBias has two occupational mentions. \enote{pd}{is
  the difference between Winobias and Winogender relevant to
  this work?}\cite{webster2018mind} released GAP, a
balanced corpus of Gendered Ambiguous Pronouns, which
measures gender bias as the ratio of F1 score on masculine
to F1 score on feminine. However the ratio is very close to 1 \shortcite{Chada_2019, Attree_2019} making it hard to compare debiasing systems. For sentiment analysis, Equity Evaluation Corpus (EEC) \shortcite{Kiritchenko_2018} was designed to measure gender bias by the difference in emotional intensity predictions between gender-swapped sentences.

An alternative way to measure gender bias is based on \textbf{association tests}, which originated from sociological research. \cite{greenwald1998measuring} proposed the Implicit Association Test (IAT) to quantify societal bias. In the IAT, response times were recorded when subjects were asked to match two concepts. For example, subjects were asked to match black and white names with ``pleasant'' and ``unpleasant'' words. Subjects tended to have shorter response times for concepts they thought associated. Based on the IAT, \cite{caliskan2017semantics} proposed the Word Embedding Association Test (WEAT), which uses word similarities between targets and attributes instead of the response times to get rid of the requirement of human subjects. \cite{may2019measuring} extended WEAT to the Sentence Embedding Association Test (SEAT); \cite{kurita2019measuring} proposed a template-based log probability bias score to measure the association between targets and attributes in BERT.

\enote{hs}{for many of the papers you discuss above it's not
  clear what the realtinship to the current work is. this
  hsould always be clear}


\subsection{Debiasing Methods}
 Many methods to remove gender bias have been proposed. The
 most common way is to define a gender direction (or, more
 generally, a subspace) by a set of gendered words, and
 debias the word embeddings in a post-processing
 projection. \cite{bolukbasi2016man} propose (i) \emph{hard
   debiasing}: they use the gendered words to compute the
 difference embedding vector as the gender direction; and
 (ii) \emph{soft debiasing},
 a
 machine learning based method
that combines
 the inner-products objective of word embedding and an
 objective to project the word embedding into an orthogonal
 gender subspace. Hard debiasing has been found to work
 better. \enote{pd}{should we mention hard-debiasing by mu
   et al here and explain the difference to bolukbasi?}
 \cite{dev2019attenuating} explored partial projection and
 some simple tricks to improve the hard debiasing
 method. \cite{zhao2019gender} applied the data
 augmentation and debiasing method of
 \cite{bolukbasi2016man} to mitigate gender bias on ELMo
 \shortcite{Peters:2018}. \cite{karve2019conceptor} introduce
 the debiasing conceptor: they shrink each
 principal component of the covariance matrix of the word
 embeddings to achieve a soft debiasing. Besides the above
 post-processing methods, \shortcite{zhao2018learning} propose
 GN-Glove: it debiases during training to learn word
 embeddings with protected attributes. The method we use
 here, DensRay, is similar to
hard debiasing in that we find
and eliminate a gender subspace in post-processing.
But DensRay can be solved efficiently in closed form and it
is more stable than hard debiasing.

 





\section{Methodology}
\subsection{Debiasing Conceptor}
\newcite{karve2019conceptor} introduced conceptor debiasing. Given a set of gendered words $V:=\{v_1,v_2,\dots,v_n\}$ and their embeddings $E$, gender bias can be mitigated by multiplying the debiasing conceptor matrix$\neg C= I-C$, where $C$ is the conceptor matrix that minimizes the objective
\begin{eqnarray}
||E-CE||^2_F+\alpha^{-2}||E||^2_F
\end{eqnarray}
where $\alpha$ is a parameter. $C$ has an analytical solution
\begin{eqnarray}
C=\frac{1}{d}EE^{T}(\frac{1}{d}EE^{T}+\alpha^{-2}I)^{-1}
\end{eqnarray}
Intuitively, C is a soft projection matrix on the linear subspace where embeddings have the maximum bias. 
Once C is computed a debiased version of embeddings $X \in R^{t \times d}$  can be obtained by matrix multiplication $(I-C)X^\intercal$.
\enote{pd}{is this correct?}

\subsection{Hard Debasing}
We will also compare with the hard debiasing method proposed
by \newcite{mu2018all}, which is originally a postprocessing
technique for improving word
representations. \newcite{karve2019conceptor} adopted it as a
method for debiasing. Hard debiasing relies upon
the assumption that the first principal component of the
embedding vectors is a meaningful gender direction. The first principal component of $E$ is the gender direction $q$.

\subsection{DensRay}
DensRay \cite{dufter2019analytical} is an analytical method for identifying the
embedding subspace of certain linguistic features. It
identifies the ``gender subspace'' using a set of gendered words
$V:=\{v_1,v_2,\dots,v_n\}$ and their embeddings $E \in
R^{n\times d}$. In contrast to the above approaches it uses a function $l$ for the gender attribute:
$l:V\to \{-1,1\}$;
e.g. $l(\mbox{father})=1$, $l(\mbox{sister})=-1$. The objective of DensRay
is to find an orthogonal matrix $Q\in R^{d\times d}$ such
that $EQ$ is  a gender subspace.

Let $L_{=}:=\{(v,w)\in V\times V|l(v)=l(w)\}$ and define
$L_{\neq}$ analogously.  The DensRay objective
in \eqref{densray1} is to maximize the distance of the word
pairs from the same gender group ($L_{=}$) and minimize the
distance of the word pairs from the different gender group
($L_{\neq}$).
\begin{eqnarray}
\max\limits_{q} 
\sum_{(v,w)\in L_{\neq}}\alpha_{\neq}||q^Td_{vw}||^2_2
-\sum_{(v,w)\in L_{=}}\alpha_{=}||q^Td_{vw}||^2_2
\eqlabel{densray1}
\end{eqnarray}
where we define $d_{vw}:=e_v-e_w$. The objective can be simplified to 
\begin{eqnarray}
\max \limits_{q} q^T(
\sum_{(v,w)\in L_{\neq}}\alpha_{\neq}||d_{vw}d_{vw}^T||^2_2-\sum_{(v,w)\in L_{=}}\alpha_{=}||d_{vw}d_{vw}^T||^2_2)q=:\max\limits_{q} q^TAq
\eqlabel{eq:densray2}
\end{eqnarray}
As stated in \cite{dufter2019analytical} $q$ is simply the eigenvector of $A$ corresponding to the largest eigenvalue.
$\alpha_{\neq},\alpha_{=}\in [0,1]$ are hyperparameters to balance the different optimization terms.

\subsection{Removing Gender Information}

Hard debiasing and DensRay yield a gender dimension $q \in R^d$. In a contextualized language model like BERT each layer yields a contextualized embedding matrix $X \in R^{t \times d}$ where $t$ is the length of the sentence. To debias representations we simply zero out the projected values on $q$ for each position, that is we set $X^{\text{debiased}}_i = X_i -  (X_i^\intercal q) q$ for each position $i$.


\subsection{Comparison of the Approaches}
\seclabel{artexample}
 \figref{fig:example} shows artificially created two dimensional embeddings. The lines show the gender directions identified by hard debiasing and DensRay. In this example
 the first principal component does not correlate with gender. DensRay is able to handle this as it explicitly uses the labels.
 
As Conceptor debiasing does not compute a single direction of gender, but always needs to apply the full conceptor matrix, it is not possible to depict the direction in the figure. 
We argue that DensRay is more interpretable than Conceptor. It allows for example to assign token level gender scores easily. In addition it affects language model performance less than Conceptor debiasing, as we show later.
 
\begin{figure}[h]
	\centering
	\includegraphics[width=0.5\linewidth]{examples.png}
	\caption{Gender direction on gendered words.}
	\figlabel{fig:example}
\end{figure}

\subsection{Adapting DensRay to Contextualized Language Models}
We now describe how we adapt DensRay to contextualized
language models. Given a set of gendered words
$V$, we extract sentences containing a word in $V$ from a
corpus. We run a contextualized language model
with $M$ layers
on a set of 
sentences that each contains one of the gender words, i.e., 
$t_1,\ldots,t_j,\ldots,t_n$ (where $t_j \in V$). 
For each $t_j $ in each sentence we  compute the contextualized representations $e_j^m, 1\leq m
\leq M$, one for each layer. We then use the vectors $e_j^m$ in \eqref{eq:densray2}.
to compute a gender direction
$q_m$ for the $m$th BERT layer. 


\section{DensRay Debiasing Experiments on BERT Layers}
\subsection{Experiments Setup}
In the experiments we use the BERT models "bert-base-uncased" and "bert-large-uncased" from the huggingface trasnformers library \citep{wolf2019huggingfaces}.

To compute the rotation matrices by DensRay, there needs a the gendered word list as label, and an input corpus to get the embeddings from BERT. For the word list, we get 23 masculine words and 23 feminine words from the "family" category\footnote{http://download.tensorflow.org/data/questions-words.txt} of the Google analogy test set \citep{mikolov2013efficient}, and label them as 1 and -1. As the input corpus,  for BERT base (large) model we collect text data from Wikipedia that contains 5,000 (10,000) occurrences of words in the gendered list, in which  the number of masculine and feminine samples are equal. The hyperparameters are set to $\alpha_{\neq}=\alpha_{=}=0.5$, since we have balanced the training samples from the corpus.

\subsection{Results on Templates}
Results about our experiments on the templates are summarized in table \ref{t:templates1}. Two example templates are given in table \ref{t:templates2}. The evaluation on our templates shows that DensRay can mitigate the gender bias on BERT.
\begin{table*}[ht]
\centering
\begin{tabular}{lllll}
\hline
model & prob(he) & prob(she) & diff & var\\
\hline
bert-base & 0.6594 & 0.1874 & 0.4720 & 0.1600 \\
bert-base-densray & 0.5106 & 0.3447 & {0.1658} & 0.0119\\
\hline
bert-large  & 0.6287 & 0.1907 & 0.4380 & 0.1262 \\
bert-large-densray  & 0.4751 & 0.2923 & {0.1827} & 0.0150\\
\hline
\end{tabular}
\caption{\label{t:templates1}
BERT debiasing results on templates. \textit{bert-base} and \textit{bert-large} are the original model without debiasing. \textit{prob(he)} is the mean probability that model predict \textit{he} as the [MASK]in all templates. \textit{var} is the variance of the differences between the probability of BERT predicts [MASK] as \textit{he} and \textit{she}.}
\end{table*}
\begin{table*}[ht]
\centering
\begin{tabular}{llll}
\hline
sentence & model & prob(he) & prob(she)\\
\hline
[MASK] is a adjunct professor. & bert-base & 0.7231 & 0.1942\\
 & bert-base-densray & 0.4423 & 0.4740\\
 & bert-large & 0.7181 & 0.2212\\
 & bert-large-densray & 0.3974 & 0.5316\\
\hline
[MASK] is a administrator. & bert-base & 0.6296 & 0.2337\\
 & bert-base-densray & 0.5045 & 0.3762\\
 & bert-large & 0.6456 & 0.2269\\
 & bert-large-densray & 0.4536 & 0.3716\\
\hline
\end{tabular}
\caption{\label{t:templates2}
Sanity check on the templates.}
\end{table*}

\subsection{Results on WEAT}
In WEAT we measure the effect size $d$-value and the oneside $p$-value of the permutation test. A higher absolute value of the $d$-value indicates larger gender bias between the target words with respect to the attribute words. So, for the $d$-value, the closer to zero, the less gender bias. Refer to the definition of the null hypothesis, if the $p$-value is less than 0.05 we will reject the null hypothesis so that there will be a significant gender bias. So, we would prefer a high $p$-value (at least 0.05) to indicate the lack of gender bias. Follow the same WEAT word lists setup as \citet{karve2019conceptor}, the results on WEAT is shown on table \ref{t:weat1}. For all the three categories, DensRay decreased absolute value of $d$-value and increased the $p$-value, although on bert-large still showd strong bias in \textit{(Career, Family) vs (Male, Female)}.
\begin{table*}[ht]
\centering
\begin{tabular}{llll}
\hline
category & model & d & p\\
\hline
(Career, Family) vs (Male, Female) & bert-base & 0.6581 & 0.08 \\
                  & bert-base-densray & 0.6397 & 0.11\\
                  & bert-large & 1.5705 & 0.00^{*} \\
                  & bert-large-densray & 0.9980 & 0.02^{*}\\
\hline
(Math, Arts) vs (Male, Female) & bert-base & 0.6017 & 0.11 \\
                  & bert-base-densray & 0.0739 & 0.45\\
                  & bert-large & 0.2239 & 0.35 \\
                  & bert-large-densray & -0.0145 & 0.48\\
\hline
(Science, Arts) vs (Male, Female) & bert-base & 0.7762 & 0.08 \\
                  & bert-base-densray & 0.0167 & 0.49\\
                  & bert-large & 0.816 & 0.04^{*}  \\
                  & bert-large-densray & 0.6743 & 0.10\\
\hline
\end{tabular}
\caption{\label{t:weat1}
BERT debiasing results on WEAT. Number with * shows significant gender bias.}
\end{table*}

\subsection{Impact on BERT Model}
Here we want to evaluate the performance of BERT model after applied DensRay. We test the perplexity of language modeling on Wikitext-2 dataset \citep{merity2016pointer} which is a subset of Wikipedia with 2 million words, the results in table \ref{t:ppl1} show that DensRay caused a small increase in perplexity on Wikitext-2 for both BERT base and large model.
\begin{table}[ht]
\centering
\begin{tabular}{llll}
\hline
model & ppl\\
\hline
bert-base & 3.7714\\
bert-base-densray & 3.8051\\
\hline
bert-large & 3.2928\\
bert-large-densray & 3.3503\\
\hline
\end{tabular}
\caption{\label{t:ppl1}
Language modeling performance on BERT after applied DensRay.}
\end{table}

Follow the same setup as \citet{wolf2019huggingfaces}\footnote{https://huggingface.co/transformers/}, we also evaluate on the GLUE tasks \citep{wang2018glue}, results are summarized in table \ref{t:glue1}. 
\begin{table*}[ht]
\centering
\begin{tabular}{llllllllll}
\hline
model & CoLA &SST-2&MRPC&STS-B&QQP&MNLI&QNLI&RTE&WNLI\\
\hline
bert-base & 3.7714& 3.7714& 3.7714& 3.7714& 3.7714& 3.7714& 3.7714& 3.7714& 3.7714\\
bert-base-densray & 3.7714& 3.7714& 3.7714& 3.7714& 3.7714& 3.7714& 3.7714& 3.7714& 3.7714\\
\hline
bert-large & 3.7714& 3.7714& 3.7714& 3.7714& 3.7714& 3.7714& 3.7714& 3.7714& 3.7714\\
bert-large-densray & 3.7714& 3.7714& 3.7714& 3.7714& 3.7714& 3.7714& 3.7714& 3.7714& 3.7714\\
\hline
\end{tabular}
\caption{\label{t:glue1}
GLUE tasks performance on BERT after applied DensRay.}
\end{table*}

\subsection{Discussions}
\subsubsection{the impact of training samples}
Through evaluation and inspection of the impact on the performance of downstream tasks, experiments show that DensRay is an effective debiasing method on BERT. Although DensRay is an analytical solution, the effect still depends on the label data. In the experiments, we regarded the occurrences of the same word in the corpus as independent words with the same gender label, and used balanced samples for masculine and feminine words. Now we analyze the impact of these processes.

Since there are only 46 words in the rendered word list, if we average their embedding under different contexts, there will be only 46 training samples left for DensRay to calculate. DensRay is essentially a supervised learning method. In the case of insufficient labels, it is difficult for supervised learning to extract useful features. Treating different occurrences as different words greatly enriches training samples. As shown in figure, the debiasing results improved when we increase the number of training samples.

The same as other projection-based debiasing methods \citep{bolukbasi2016man,zhao2019gender,dev2019attenuating, karve2019conceptor}, the premise of DensRay debiasing is that the bias direction should be correct. If the sample is unbalanced, the bias direction calculated by DensRay will be biased towards either the male or the female, resulting in deleting the gender subspace during debiasing will reverse the gender bias (e.g. there are more masculine words in unbalanced text data, thus the embeddings will be biased towards female after biased).The figure also shows that balanced training sample improved the debiasing performers. 
\begin{figure*}
    \centering
    \includegraphics{}
    \caption{Here should be a graph.}
    \label{fig:my_label}
\end{figure*}

\subsubsection{the ways of debiasing}
In this experiment, we used the method of removing the first dimension (replacing its value by $0$) of the gender interpreteble subspace to remove gender bias. Here we explore some other ways.

In figure \ref{fig:meandebias}, we tried two other ways to remove bias. The first is to replace the first dimension of the gender interpreteble subspace with the mean value of the first dimension of the training samples. The second way is to standardize the first dimension. The results showed that both of these methods did not perform well. We further checked the mean and found that the mean of the different layers is not stable around 0, which is a problem worthy for further exploring.
\begin{figure*}
    \centering
    \includegraphics{}
    \caption{Here should be a graph.}
    \label{fig:meandebias}
\end{figure*}

As shown in figure \ref{moredim}, we try to delete more dimensions. The results show that removing more dimensions does not improve the debiasing results significantly.
\begin{figure*}
    \centering
    \includegraphics{}
    \caption{Here should be a graph.}
    \label{fig:moredim}
\end{figure*}

\subsubsection{debiasing on each BERT layer}
Here we only apply DesnRay on one BERT layer at a time. We constructed a table to illustrate the top three layers with the best performance on our templates and the three WEAT categories.
\begin{table*}[ht]
\centering
\begin{tabular}{llllllllll}
\hline
\end{tabular}
\caption{\label{t:bestlayers}
Here needs a table.}
\end{table*}



\section{DensRay Debiasing multilingual-BERT}
\subsection{Setup}
As an extension, we apply DensRay to mBERT for zero-shot debiasing on Chinese. Here we use the "bert-multilingual-uncased" model from \citep{wolf2019huggingfaces}, we also use the same setup as the "bert-base-uncased" model in our previous experiments.

As before, we compute the rotation matrices using the English gendered words from the "family" category of the Google analogy test set \citep{mikolov2013efficient}.

Since Chinese is a language that does not contain genus, we can construct the OCCTMP templates by directly translating from the English templates. So we got the following form: \eat{"\text{[MASK]}是一个\textit{occupation}。".} For the occupation name, we referred to Tencent Translation\footnote{https://fanyi.qq.com/} and made some manual adjustments to the translation. After removing the duplicates, we got 302 Chinese templates.

\subsection{Results on OCCTMP}
Results about our experiments on the templates are summarized in \tabref{t:templates3}. Two examples are given in \tabref{t:templates3}. It shows that DensRay trained with English can mitigate gender bias in mBERT: the average difference drops from 0.17 to 0.08 on Chinese templates. Also, the mBERT still gets comparable perplexities on Wikitext-2 after debiasing, see table \tabref{t:ppl2}. 
\begin{table}[ht]
\centering
\footnotesize
\begin{tabular}{lcccc}
\hline
model & prob(he) & prob(she) & diff & var\\
\hline
\scriptsize bert-multi-en 
& 0.51 & 0.14 & 0.36 & 0.06 \\
\scriptsize 
bert-multi-densray-en & 0.33 & 0.12 & 0.21 & 0.03 \\
\scriptsize bert-multi-cn 
& 0.24 & 0.07 & 0.17 & 0.02 \\
\scriptsize bert-multi-densray-cn 
& 0.12 & 0.04 & 0.08 & 0.01\\
\hline
\end{tabular}
\caption{\tablabel{t:templates3}
Results of OCCTMP on mBERT after applied DensRay. Models with \textit{-en} are tested on English templates, and those with \textit{-cn} are tested on Chinese templates.}
\end{table}
\begin{table}[ht]
\centering
\footnotesize
\begin{tabular}{lc}
\hline
model & ppl\\
\hline
bert-multi & 3.58\\
bert-multi-densray & 3.72\\
\hline
\end{tabular}
\caption{\tablabel{t:ppl2}
Language modeling performance on mBERT after applied DensRay.}
\end{table}
\begin{table}[t]
\centering
\footnotesize
\begin{tabular}{llcc}
\hline
sentence & model & \eat{prob(他)} & \eat{prob(她)}\\
\hline
\eat{\text{[MASK]}是一个客座教授。} & bert-multi-en & 0.68 & 0.16\\
& bert-multi-densray-en & 0.51 & 0.18\\
& bert-multi-cn & 0.52 & 0.11\\
 & bert-multi-densray-cn & 0.30 & 0.08\\
\hline
\eat{\text{[MASK]}是一个管理员。} & bert-multi-en & 0.53 & 0.17\\
& bert-multi-densray-en & 0.35 & 0.13\\
& bert-multi-cn & 0.68 & 0.16\\
 & bert-multi-densray-cn & 0.51 & 0.18\\
\hline
\end{tabular}
\caption{\label{t:templates3}
Sanity check on the Chinese templates, where \eat{\textit{他}} means \textit{he} and \eat{\textit{她}} means \textit{she}. The two sentences are translated from \tabref{t:templates2}.}
\end{table}

\section{Conclusion}
We introduced DensRay debiasing on BERT. Our experiments
show that this method can effectively mitigate gender bias on OCCTMP and the Association Tests, while maintained the performance of BERT on language modeling and GLUE tasks. We applied DensRay to BERT attention heads, showed that gender information is processed in all attention heads, there is no single attention head responsible for processing gender information. We also used DensRay to obtain interpretable gender scores, to quantify bias on token and sentence level for all representations. Finally, we demonstrated that we can remove bias multilingually, we used only English training data to effectively debias Chinese. \eat{As to further research, we plan to investigate other linguistic features on multilingual spaces by DensRay.}


\enote{hs}{three parts of the paper should be in sync:

  (i) the abstract

  (ii) the contributions at the end of the introduciton

  (iii) the conclusino

  make sure to check that during final editing}



%\section*{Acknowledgments}
%
%The acknowledgments should go immediately before the references. Do not number the acknowledgments %section.
%Do not include this section when submitting your paper for review.

\bibliographystyle{acl_natbib}
\bibliography{anthology,emnlp2020}

%\appendix
%\section{Appendices}
%\section{Supplemental Material}
\end{document}
